%%%%%%%%%%%%%%%%%%%%%%%%%%%%%%%%%%%%%%%%%
% Formalizing Group Action in Lean
% LaTeX Beamer Presentation
% MATH70040 Formalising Mathematics
%
%!TEX program = xelatex
%%%%%%%%%%%%%%%%%%%%%%%%%%%%%%%%%%%%%%%%%

%----------------------------------------------------------------------------------------
%	CLASS, PACKAGES AND OTHER DOCUMENT CONFIGURATIONS
%----------------------------------------------------------------------------------------

\documentclass[
	aspectratio=169,
	t,
	onlytextwidth,
	10pt,
]{beamer}

\usetheme{Imperial}

% Override monospaced font to support Unicode math symbols in code
% (e.g., ∀, ∃, →) that appear in Lean snippets
% Reset fontspec Path so system fonts can be found
\defaultfontfeatures{Ligatures=TeX,Path=}
\IfFontExistsTF{FreeMono}{
  \setmonofont{FreeMono}[Scale=MatchLowercase]
}{
  \setmonofont{Imperial Sans Text}[Scale=MatchLowercase]
}

%----------------------------------------------------------------------------------------
%	REQUIRED PACKAGES FOR CODE AND MATH
%----------------------------------------------------------------------------------------

% Code highlighting with minted for Lean4
\usepackage{minted}
\usemintedstyle{pastie}
\setminted{
	fontsize=\small,
	linenos=true,
	python3=true,
	breaklines=true,
	frame=lines,
}

% Math packages
\usepackage{amsmath}
\usepackage{amssymb}
\usepackage{mathtools}

% Additional utilities
\usepackage{booktabs}
\usepackage{tabularx}
\usepackage{changepage}

% For code snippets
% (listings removed; minted is used for Lean code blocks)

%----------------------------------------------------------------------------------------
%	CUSTOM COMMANDS FOR PRESENTATIONS
%----------------------------------------------------------------------------------------

% GitHub URL helper for code references
\newcommand{\githuburl}[2]{%
	\href{https://github.com/ffwcheng/group-action/tree/#1/src#2}{\texttt{#2}}%
}

% Lean code inline
\newcommand{\leancode}[1]{\texttt{\small #1}}

% Two-column layout helper (math vs code)
\newcommand{\mathvscode}[4]{
	\begin{columns}[T]
		\begin{column}{0.48\linewidth}
			\textbf{Mathematics}\\[0.3em]
			#1
		\end{column}
		\begin{column}{0.48\linewidth}
			\textbf{Lean 4 Code}\\[0.3em]
			#2
		\end{column}
	\end{columns}
}

% Theorem definition
\newcommand{\thmbox}[2]{
	\fbox{\begin{minipage}{\dimexpr\linewidth-2\fboxsep-2\fboxrule}
	\textbf{#1}\\[0.5em]
	#2
	\end{minipage}}
}

%----------------------------------------------------------------------------------------
%	PRESENTATION METADATA
%----------------------------------------------------------------------------------------

\title{Formalizing Group Action in Lean}
\subtitle{Project 1: Group Actions in Lean 4}
\author{Frankie Feng-Cheng WANG}
\institute{Department of Mathematics, Imperial College London}
\date{MATH70040 Formalising Mathematics}

% Additional metadata for title slide
\newcommand{\course}{MATH70040 Formalising Mathematics}
\newcommand{\githublink}{https://github.com/FrankieeW/GroupAction}

%----------------------------------------------------------------------------------------
%	DOCUMENT START
%----------------------------------------------------------------------------------------

\begin{document}

%----------------------------------------------------------------------------------------
%	TITLE SLIDE (ICLBlue background)
%----------------------------------------------------------------------------------------

\begingroup
\setbeamercolor{background canvas}{bg=ICLBlue}
\setbeamercolor{title page title}{fg=white}
\setbeamercolor{title page subtitle}{fg=white}
\setbeamercolor{author}{fg=white}
\setbeamercolor{inst}{fg=white}
\setbeamercolor{date}{fg=white}
\setbeamertemplate{title page}[logo]{ICL_Logo_White.pdf}
\frame[plain, s]{\titlepage}
\endgroup

%----------------------------------------------------------------------------------------
%	AGENDA SLIDE
%----------------------------------------------------------------------------------------

\begingroup
\setbeamercolor{background canvas}{bg=ICLBlue}
\setbeamercolor{normal text}{fg=white}\usebeamercolor[fg]{normal text}
\setbeamercolor{page number in head/foot}{fg=white}

\begin{frame}
	\frametitle{\textcolor{white}{Outline}}
	\begin{columns}[T]
		\begin{column}{0.95\linewidth}
			\Large
			\vspace{0.5em}
			\textbf{1. Introduction}\\
			\quad Motivation \& Main Results\\[0.8em]

			\textbf{2. Core Definitions}\\
			\quad GroupAction, Faithful, Transitive\\[0.8em]

			\textbf{3. Examples}\\
			\quad Symmetric Group, $D_4$, and more\\[0.8em]

			\textbf{4. Key Theorems}\\
			\quad Theorem 16.3 (Permutation Representation) \& Theorem 16.12 (Stabilizer Subgroup)\\[0.8em]

			\textbf{5. Reflection}\\
			\quad Challenges \& Future Extensions
		\end{column}
	\end{columns}
\end{frame}
\endgroup

%----------------------------------------------------------------------------------------
%	SECTION 1: INTRODUCTION (2 slides)
%----------------------------------------------------------------------------------------

\begin{frame}
	\frametitle{What is a Group Action?}

	\textbf{Motivation:}
	\begin{itemize}
		\item Groups act on sets, formalizing \textit{symmetry}
		\item Connects abstract algebra with concrete transformations
		\item Foundation for orbit-stabilizer theorems, Burnside's lemma, etc.
	\end{itemize}

	\vspace{1em}

	\textbf{Project Goals:}
	\begin{itemize}
		\item[\checkmark] Core Definitions: \leancode{GroupAction} typeclass, Faithful, Transitive
		\item[\checkmark] Concrete Examples: Symmetric group, $D_4$, conjugation, etc.
		\item[\checkmark] Key Theorems: Permutation representation (Thm 16.3), Stabilizer subgroup (Thm 16.12)
	\end{itemize}

\end{frame}

%------------------------------------------------

\begin{frame}
	\frametitle{Main Results}

	\vspace{0.5em}

	\thmbox{Theorem 16.3: Permutation Representation}{%
		Every group action induces a group homomorphism $\phi: G \to \mathrm{Sym}(X)$
	}

	\vspace{1.5em}

	\thmbox{Theorem 16.12: Stabilizer Subgroup}{%
		For any $x \in X$, the stabilizer $G_x = \{g \in G \mid g \cdot x = x\}$ is a subgroup of $G$
	}

	\vspace{1.5em}

	\small\textit{Both theorems are fully formalized in Lean 4 with explicit proofs.}

\end{frame}

%----------------------------------------------------------------------------------------
%	SECTION 2: DEFINITIONS (3 slides)
%----------------------------------------------------------------------------------------

\begin{frame}[fragile]
	\frametitle{Group Action Definition}
	\framesubtitle{Mathematical Foundation}

	\begin{columns}[T]
		\begin{column}{0.48\linewidth}
			\textbf{Mathematics}\\[0.5em]

			A group $G$ \textbf{acts} on a set $X$ via a function
			\[ \cdot: G \times X \to X \]

			\textbf{Axioms:}
			\begin{enumerate}
				\item \textbf{Associativity}: $(g_1 g_2) \cdot x = g_1 \cdot (g_2 \cdot x)$
				      \smallskip
				\item \textbf{Identity}: $1 \cdot x = x$
			\end{enumerate}

			\vspace{0.5em}
			for all $g_1, g_2 \in G, x \in X$
		\end{column}
		\begin{column}{0.48\linewidth}
			\textbf{Lean 4 Code}\\[0.5em]

			\begin{minted}[fontsize=\tiny,linenos=false]{lean}
class GroupAction (G : Type*) 
    [Monoid G] (X : Type*) where
  act : G → X → X
  ga_mul : ∀ g₁ g₂ x, 
    act (g₁ * g₂) x = 
    act g₁ (act g₂ x)
  ga_one : ∀ x, act 1 x = x
			\end{minted}

			\vspace{0.3em}
			{\tiny Source: \githuburl{v1.2.1-lean-only}{/Defs.lean:15-22}}
		\end{column}
	\end{columns}

\end{frame}

%------------------------------------------------

\begin{frame}[fragile]
	\frametitle{Orbits and Stabilizers}

	% TODO: Define orbit and stabilizer sets
	% - Orbit: Orb(x) = {g · x | g ∈ G}
	% - Stabilizer: Stab(x) = {g ∈ G | g · x = x}
	% - Lean code for these definitions
	% - Connection to equivalence classes

\end{frame}

%------------------------------------------------

\begin{frame}[fragile]
	\frametitle{Faithful Actions}

	\begin{columns}[T]
		\begin{column}{0.48\linewidth}
			\textbf{Mathematics}\\[0.5em]

			An action is \textbf{faithful} if distinct group elements act differently.

			\vspace{0.8em}

			\textbf{Formally:}
			\[ \forall g_1, g_2 \in G, \, (\forall x, g_1 \cdot x = g_2 \cdot x) \Rightarrow g_1 = g_2 \]

			\vspace{0.8em}

			\textbf{Intuition:} The action ``faithfully represents'' the group structure
		\end{column}
		\begin{column}{0.48\linewidth}
			\textbf{Lean 4 Code}\\[0.5em]

			\begin{minted}[fontsize=\tiny,linenos=false]{lean}
def GroupAction.faithful 
    {G : Type*} [Group G] 
    {X : Type*} 
    [GroupAction G X] : Prop :=
  ∀ g₁ g₂ : G,
    (∀ x : X, 
      GroupAction.act g₁ x = 
      GroupAction.act g₂ x) → 
    g₁ = g₂
			\end{minted}

			\vspace{0.3em}
			{\tiny Source: \githuburl{v1.2.1-lean-only}{/Defs.lean:26-28}}
		\end{column}
	\end{columns}

\end{frame}

%------------------------------------------------

\begin{frame}[fragile]
	\frametitle{Transitive Actions}

	\begin{columns}[T]
		\begin{column}{0.48\linewidth}
			\textbf{Mathematics}\\[0.5em]

			An action is \textbf{transitive} if any element can be moved to any other.

			\vspace{0.8em}

			\textbf{Formally:}
			\[ \forall x_1, x_2 \in X, \exists g \in G, \, g \cdot x_1 = x_2 \]

			\vspace{0.8em}

			\textbf{Intuition:} The group ``acts transitively'' on the entire set
		\end{column}
		\begin{column}{0.48\linewidth}
			\textbf{Lean 4 Code}\\[0.5em]

			\begin{minted}[fontsize=\tiny,linenos=false]{lean}
def GroupAction.transitive 
    {G : Type*} [Group G] 
    {X : Type*} 
    [GroupAction G X] : Prop :=
  ∀ x₁ x₂ : X, 
    ∃ g : G, 
      GroupAction.act g x₁ = x₂
			\end{minted}

			\vspace{0.3em}
			{\tiny Source: \githuburl{v1.2.1-lean-only}{/Defs.lean:33-35}}
		\end{column}
	\end{columns}

\end{frame}

%----------------------------------------------------------------------------------------
%	SECTION 3: EXAMPLES (7 slides)
%----------------------------------------------------------------------------------------

\begin{frame}[fragile]
	\frametitle{Example 1: Symmetric Group on $X$}

	$\mathrm{Sym}(X)$ acts on $X$ by applying permutations.

	\vspace{0.8em}

	\begin{minted}[fontsize=\tiny,linenos=false]{lean}
instance permGroupAction (X : Type*) : GroupAction (Equiv.Perm X) X :=
  { act := fun g x => g x
    ga_mul := by intro g1 g2 x; rfl
    ga_one := by intro x; rfl }
	\end{minted}

	\vspace{0.8em}

	\textit{One of the most fundamental actions.}

	{\tiny Source: \githuburl{v1.2.1-lean-only}{/Examples.lean:20-27}}

\end{frame}

%------------------------------------------------

\begin{frame}[fragile]
	\frametitle{Example 1: Symmetric Group Properties}
	\framesubtitle{Faithful \& Transitive}

	\begin{columns}[T]
		\begin{column}{0.48\linewidth}
			\textbf{Faithful}\\[0.3em]

			\begin{minted}[fontsize=\tiny,linenos=false]{lean}
theorem permFaithful (X : Type*) [Nonempty X] :
    faithful (Equiv.Perm X) X := by
  intro g1 g2 h
  ext x
  have := h x
  exact this
			\end{minted}
		\end{column}
		\begin{column}{0.48\linewidth}
			\textbf{Transitive}\\[0.3em]

			\begin{minted}[fontsize=\tiny,linenos=false]{lean}
theorem permTransitive (X : Type*) [Nonempty X] :
    transitive (Equiv.Perm X) X := by
  intro x y
  obtain ⟨z⟩ := ‹Nonempty X›
  use Equiv.Perm.swap z x |>.trans 
       (Equiv.Perm.swap z y)
  simp
			\end{minted}
		\end{column}
	\end{columns}

	{\tiny Source: \githuburl{v1.2.1-lean-only}{/Examples.lean:29-44}}

\end{frame}

%------------------------------------------------

\begin{frame}[fragile]
	\frametitle{Example 2: Left Multiplication}

	$G$ acts on itself by left multiplication: $g_1 \cdot g_2 = g_1 * g_2$

	\vspace{0.8em}

	\begin{minted}[fontsize=\tiny,linenos=false]{lean}
instance groupAsGSet (G : Type*) [Group G] : GroupAction G G :=
  { act := fun g1 g2 => g1 * g2
    ga_mul := by intro g1 g2 g3; rw [mul_assoc]
    ga_one := by intro g; rw [one_mul] }
	\end{minted}

	\vspace{0.8em}

	\small This action is \textbf{transitive} but \textbf{not faithful}.

	{\tiny Source: \githuburl{v1.2.1-lean-only}{/Examples.lean:46-54}}

\end{frame}

%------------------------------------------------

\begin{frame}[fragile]
	\frametitle{Example 3: Subgroup Action}

	A subgroup $H \leq G$ acts on $G$ by left multiplication.

	\vspace{0.8em}

	\begin{minted}[fontsize=\tiny,linenos=false]{lean}
instance subgroupAsGSet (G : Type*) [Group G] (H : Subgroup G) :
    GroupAction H G :=
  { act := fun ⟨h, _⟩ g => h * g
    ga_mul := by intro ⟨h1,_⟩ ⟨h2,_⟩ g; simp; rw [mul_assoc]
    ga_one := by intro g; simp; rw [one_mul] }
	\end{minted}

	\vspace{0.8em}

	\small Coercion from \leancode{H} to \leancode{G} handled implicitly.

		{\tiny Source: \githuburl{v1.2.1-lean-only}{/Examples.lean:58-65}}

\end{frame}

%------------------------------------------------

\begin{frame}[fragile]
	\frametitle{Example 4: Conjugation}

	Conjugation: $h \cdot g = h g h^{-1}$

	\vspace{0.6em}

	\begin{minted}[fontsize=\tiny,linenos=false]{lean}
instance conjugationAction (G : Type*) [Group G] : GroupAction G G :=
  { act := fun g h => g * h * g⁻¹
    ga_mul := fun g1 g2 g => by
      calc g1 * g2 * g * (g1 * g2)⁻¹
           = g1 * g2 * g * g2⁻¹ * g1⁻¹ := by rw [mul_inv_rev]
           _ = g1 * (g2 * g * g2⁻¹) * g1⁻¹ := by rw [mul_assoc, mul_assoc]
    ga_one := fun g => by simp }
	\end{minted}

	\vspace{0.6em}

	\small Orbits are \textbf{conjugacy classes}; stabilizers are \textbf{centralizers}.

	{\tiny Source: \githuburl{v1.2.1-lean-only}{/Examples.lean:67-78}}

\end{frame}

%------------------------------------------------

\begin{frame}[fragile]
	\frametitle{Example 5: Scalar Action on $\mathbb{C}^n$}

	$\mathbb{C}^\times$ acts on $\mathbb{C}^n$ by componentwise multiplication.

	\vspace{0.6em}

	\begin{minted}[fontsize=\tiny,linenos=false]{lean}
instance complexScalarAction (n : ℕ) : 
    GroupAction ℂ* (Fin n → ℂ) :=
  { act := fun ⟨c, _⟩ v => fun i => c * v i
    ga_mul := by intro ⟨c1,_⟩ ⟨c2,_⟩ v; ext i; ring
    ga_one := by intro v; ext i; simp }
	\end{minted}

	\vspace{0.6em}

	\small Uses \leancode{Fin n → ℂ} to represent $\mathbb{C}^n$ in Lean.

		{\tiny Source: \githuburl{v1.2.1-lean-only}{/Examples.lean:82-92}}

\end{frame}

%------------------------------------------------

\begin{frame}[fragile]
	\frametitle{Example 6: Dihedral Group $D_4$}

	$D_4$ (symmetries of a square) acts on $\mathbb{Z}/4$ (vertices).

	\vspace{0.4em}

	\begin{minted}[fontsize=\tiny,linenos=false]{lean}
abbrev D4 := DihedralGroup 4

def d4Act : D4 → (Fin 4) → (Fin 4)
  | .r k, v => ((v.val + k) % 4 : Fin 4)
  | .sr k, v => ((k - v.val) % 4 : Fin 4)

instance d4ActionZMod4 : GroupAction D4 (Fin 4) :=
  { act := d4Act
    ga_mul := by intro g1 g2 v; 
      cases g1 <;> cases g2 <;> simp [d4Act]; ring_nf
    ga_one := by intro v; simp [d4Act] }
	\end{minted}

	\vspace{0.4em}

	\small \textbf{Geometric interpretation}: Rotations and reflections of square vertices.

		{\tiny Source: \githuburl{v1.2.1-lean-only}{/Examples.lean:108-156}}

\end{frame}

%----------------------------------------------------------------------------------------
%	SECTION 4: ORBITS & STABILIZERS (2 slides)
%----------------------------------------------------------------------------------------

\begin{frame}[fragile]
	\frametitle{Orbit-Stabilizer Theorem Setup}
	\framesubtitle{Relating |G|, |Orb(x)|, |Stab(x)|}

	% TODO: State and explain the theorem
	% - Theorem 16.3: |G| = |Orb(x)| · |Stab(x)|
	% - Bijection between Stab(x) and G/Stab(x)
	% - Lean type-theoretic formulation

\end{frame}

%------------------------------------------------

\begin{frame}[fragile]
	\frametitle{Burnside's Lemma Application}
	\framesubtitle{Counting Fixed Points}

	% TODO: Number of orbits formula
	% - Formula: # orbits = (1/|G|) Σ_{g∈G} |Fix(g)|
	% - Symmetry counting problem
	% - Lean proof sketch or reference

\end{frame}

%----------------------------------------------------------------------------------------
%	SECTION 5: THEOREM 16.3 - Orbit-Stabilizer (5 slides)
%----------------------------------------------------------------------------------------

\begin{frame}[fragile]
	\frametitle{Theorem 16.3: Permutation Representation}

	\begin{block}{Theorem 16.3}
		Every group action induces a group homomorphism $\phi: G \to \mathrm{Sym}(X)$ such that:
		\[ \phi(g)(x) = g \cdot x \quad \text{for all } g \in G, x \in X \]
	\end{block}

	\vspace{1em}

	\textit{This theorem connects group actions with permutation representations.}

\end{frame}

%------------------------------------------------

\begin{frame}[fragile]
	\frametitle{Proof Strategy: 5 Steps}

	\begin{enumerate}
		\item \textbf{Define $\sigma_g$}: For each $g \in G$, construct $\sigma_g: X \to X$

		\item \textbf{Prove Bijection}: Show $\sigma_g$ is bijective (left/right inverse)

		\item \textbf{Construct $\phi$}: Package $\sigma_g$ as $\phi(g) \in \mathrm{Sym}(X)$

		\item \textbf{Verify Homomorphism}: Prove $\phi(g_1 g_2) = \phi(g_1) \circ \phi(g_2)$

		\item \textbf{Package Theorem}: Combine all pieces into final statement
	\end{enumerate}

\end{frame}

%------------------------------------------------

\begin{frame}[fragile]
	\frametitle{Step 1-2: Define $\sigma_g$ \& Prove Bijection}

	\begin{columns}[T]
		\begin{column}{0.48\linewidth}
			\textbf{Define $\sigma_g$}\\[0.5em]

			\begin{minted}[fontsize=\tiny,linenos=false]{lean}
def sigma (g : G) : X → X :=
  fun x => 
    GroupAction.act g x
			\end{minted}
		\end{column}
		\begin{column}{0.48\linewidth}
			\textbf{Prove Bijective}\\[0.5em]

			\begin{minted}[fontsize=\tiny,linenos=false]{lean}
def sigmaPerm (g : G) : 
    Equiv.Perm X :=
  { toFun := sigma g
    invFun := sigma g⁻¹
    left_inv := by
      intro x
      calc GroupAction.act g⁻¹ 
           (GroupAction.act g x)
        = GroupAction.act 
          (g⁻¹ * g) x := by
          rw [←ga_mul]
        _ = x := by simp
    right_inv := ... }
			\end{minted}
		\end{column}
	\end{columns}

	\vspace{0.8em}

	\small \textbf{Key insight:} $g^{-1}$ provides the inverse map.

		{\tiny Source: \githuburl{v1.2.1-lean-only}{/Permutation.lean:24-67}}

\end{frame}

%------------------------------------------------

\begin{frame}[fragile]
	\frametitle{Step 3-4: Construct $\phi$ \& Verify Homomorphism}

	\textbf{Define $\phi: G \to \mathrm{Sym}(X)$}\\[0.5em]

	\begin{minted}[fontsize=\tiny,linenos=false]{lean}
def phi : G → Equiv.Perm X :=
  fun g => sigmaPerm g
	\end{minted}

	\vspace{1em}

	\textbf{Prove $\phi(g_1 g_2) = \phi(g_1) \circ \phi(g_2)$}\\[0.5em]

	\begin{minted}[fontsize=\tiny,linenos=false]{lean}
lemma phi_mul (g₁ g₂ : G) : 
    phi (g₁ * g₂) = phi g₁ * phi g₂ := by
  apply Equiv.ext
  intro x
  calc phi (g₁ * g₂) x 
      = GroupAction.act (g₁ * g₂) x := rfl
    _ = GroupAction.act g₁ 
        (GroupAction.act g₂ x) := 
        GroupAction.ga_mul g₁ g₂ x
	\end{minted}

	\vspace{0.8em}

	\small \textbf{Highlight:} Uses \leancode{GroupAction.ga_mul} axiom for associativity.

		{\tiny Source: \githuburl{v1.2.1-lean-only}{/Permutation.lean:69-93}}

\end{frame}

%------------------------------------------------

\begin{frame}[fragile]
	\frametitle{Step 5: Package the Theorem}

	\textbf{Final Lean Theorem}\\[0.8em]

	\begin{minted}[fontsize=\tiny,linenos=false]{lean}
theorem group_action_to_perm_representation :
  ∃ (φ : G → Equiv.Perm X),
    (∀ g x, φ g x = GroupAction.act g x) ∧
    (∀ g₁ g₂, φ (g₁ * g₂) = φ g₁ * φ g₂) ∧
    (φ 1 = 1) := by
  exact ⟨phi, ⟨phi_apply, ⟨phi_mul, phi_one⟩⟩⟩
	\end{minted}

	\vspace{1em}

	\textbf{Proof by construction:} We exhibit $\phi$ and verify all properties.
	\begin{itemize}
		\item Action property: $\phi(g)(x) = g \cdot x$
		\item Homomorphism property: $\phi(g_1 g_2) = \phi(g_1) \circ \phi(g_2)$
		\item Identity property: $\phi(1) = \text{id}$
	\end{itemize}

	{\tiny Source: \githuburl{v1.2.1-lean-only}{/Permutation.lean:102-114}}

\end{frame}

%----------------------------------------------------------------------------------------
%	SECTION 6: THEOREM 16.12 (2 slides)
%----------------------------------------------------------------------------------------

\begin{frame}[fragile]
	\frametitle{Theorem 16.12: Stabilizer Subgroup}

	\begin{columns}[T]
		\begin{column}{0.48\linewidth}
			\textbf{Mathematics}\\[0.5em]

			\textbf{Definition:} The \textbf{stabilizer} of $x \in X$ is:
			\[ G_x = \{g \in G \mid g \cdot x = x\} \]

			\vspace{1em}

			\textbf{Theorem 16.12:} $G_x$ is a subgroup of $G$

			\vspace{0.8em}

			\textbf{Proof requires:}
			\begin{enumerate}
				\item $1 \in G_x$ (identity)
				\item $g_1, g_2 \in G_x \Rightarrow g_1 g_2 \in G_x$ (closure)
				\item $g \in G_x \Rightarrow g^{-1} \in G_x$ (inverses)
			\end{enumerate}
		\end{column}
		\begin{column}{0.48\linewidth}
			\textbf{Lean Structure}\\[0.5em]

			\begin{minted}[fontsize=\tiny,linenos=false]{lean}
def stabilizerSet (x : X) : 
    Set G :=
  { g : G | 
    GroupAction.act g x = x }

def stabilizer (x : X) : 
    Subgroup G := 
  { carrier := stabilizerSet x
    one_mem' := ...
    mul_mem' := ...
    inv_mem' := ... }
			\end{minted}

			\vspace{0.5em}

			\small Lean requires explicit proofs of all three subgroup axioms.
		\end{column}
	\end{columns}

	{\tiny Source: \githuburl{v1.2.1-lean-only}{/Stabilizer.lean:22-56}}

\end{frame}

%------------------------------------------------

\begin{frame}[fragile]
	\frametitle{Stabilizer in Lean: Definition \& Proof}

	\textbf{Part 1: Define Stabilizer Set} (Stabilizer.lean:22-23)\\[0.5em]

	\begin{minted}[fontsize=\tiny,linenos=false]{lean}
def stabilizerSet (x : X) : Set G :=
  { g : G | GroupAction.act g x = x }
	\end{minted}

	\vspace{1em}

	\textbf{Part 2: Construct Subgroup} (Stabilizer.lean:26-54, key excerpts)\\[0.5em]

	\begin{minted}[fontsize=\tiny,linenos=false]{lean}
def stabilizer (x : X) : Subgroup G := by
  exact
    { carrier := stabilizerSet x
      one_mem' := by simp [stabilizerSet, GroupAction.ga_one x]
      mul_mem' := by
        intro g₁ g₂ hg₁ hg₂
        calc GroupAction.act (g₁ * g₂) x
            = GroupAction.act g₁ (GroupAction.act g₂ x) := by 
              simp using (GroupAction.ga_mul g₁ g₂ x)
          _ = GroupAction.act g₁ x := by rw [hg₂]
          _ = x := hg₁
      inv_mem' := by
        intro g hg
        calc GroupAction.act g⁻¹ x
            = GroupAction.act g⁻¹ (GroupAction.act g x) := by rw [hg]
          _ = x := by simp [GroupAction.ga_mul, GroupAction.ga_one] }
	\end{minted}

	\vspace{0.8em}

	\small \textbf{Highlight:} Each subgroup axiom proven with \leancode{calc} proofs.

		{\tiny Source: \githuburl{v1.2.1-lean-only}{/Stabilizer.lean:22-56}}

\end{frame}

%----------------------------------------------------------------------------------------
%	SECTION 7: REFLECTION (3 slides)
%----------------------------------------------------------------------------------------

\begin{frame}
	\frametitle{What Lean Guarantees}

	\begin{itemize}
		\item[\checkmark] \textbf{Type Correctness}: All functions type-check, no runtime type errors

		\item[\checkmark] \textbf{No Hidden Assumptions}: Every axiom explicitly declared

		\item[\checkmark] \textbf{Axiom Matching}: Our proofs use only standard mathlib axioms (no choice beyond mathlib)

		\item[\checkmark] \textbf{Subgroup Verification}: \leancode{Stabilizer} is proven to satisfy all subgroup axioms

		\item[\checkmark] \textbf{Homomorphism Properties}: $\phi$ proven to preserve group structure
	\end{itemize}

	\vspace{1.5em}

	\textit{Lean's proof assistant guarantees mathematical correctness—no gaps, no handwaving.}

\end{frame}

%------------------------------------------------

\begin{frame}
	\frametitle{Challenges \& Lessons Learned}

	\textbf{Technical Challenges:}
	\begin{itemize}
		\item \textbf{Typeclass Resolution}: Manual instance declarations for \leancode{GroupAction}
		\item \textbf{Equiv Mechanism}: Understanding \leancode{Equiv.Perm} vs raw bijections
		\item \textbf{Coercions}: Handling coercion from \leancode{Subgroup H} to \leancode{H} in examples
		\item \textbf{Function Extensionality}: Using \leancode{Equiv.ext} to prove permutation equality
	\end{itemize}

	\vspace{1em}

	\textbf{Lessons:}
	\begin{itemize}
		\item Read Mathlib source code for patterns
		\item Use \leancode{calc} mode for clarity
		\item Lean enforces rigor: every step must be justified
	\end{itemize}

\end{frame}

%------------------------------------------------

\begin{frame}
	\frametitle{Conclusion \& Future Work}

	\textbf{What We Achieved:}
	\begin{itemize}
		\item Formalized core group action theory in Lean 4
		\item 7 concrete examples from abstract algebra
		\item 2 fundamental theorems with complete proofs
	\end{itemize}

	\vspace{1.5em}

	\textbf{Future Extensions:}
	\begin{itemize}
		\item \textbf{Orbit-Stabilizer Theorem}: $|G| = |\text{Orb}(x)| \cdot |G_x|$
		\item \textbf{Burnside's Lemma}: Counting orbits under symmetry
		\item \textbf{Applications}: Cayley's Theorem, Sylow Theorems
		\item \textbf{Category Theory}: Generalizations to functors and natural transformations
	\end{itemize}

\end{frame}

%----------------------------------------------------------------------------------------
%	THANK YOU SLIDE
%----------------------------------------------------------------------------------------

\begingroup
\setbeamercolor{background canvas}{bg=ICLBlue}
\setbeamercolor{normal text}{fg=white}\usebeamercolor[fg]{normal text}
\setbeamercolor{page number in head/foot}{fg=white}

\begin{frame}[plain]
	\vfill
	\centering

	{\Huge\textbf{Thank You!}}

	\vspace{2em}

	{\large\textit{Questions?}}

	\vspace{2em}

	\hrule

	\vspace{2em}

	{\large\textbf{GitHub Repository}}\\[0.5em]
	\small\url{https://github.com/FrankieeW/GroupAction}\\
	Version: \texttt{v1.2.1-lean-only} (stable)

	\vspace{2em}

	{\large\textbf{Contact}}\\[0.5em]
	\small Frankie Feng-Cheng WANG\\
	Department of Mathematics, Imperial College London\\
	MATH70040 Formalising Mathematics

	\vfill
\end{frame}
\endgroup

%----------------------------------------------------------------------------------------

\end{document}
