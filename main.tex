%%%%%%%%%%%%%%%%%%%%%%%%%%%%%%%%%%%%%%%%%
% Formalizing Group Action in Lean
% LaTeX Beamer Presentation
% MATH70040 Formalising Mathematics
%
%!TEX program = xelatex
%%%%%%%%%%%%%%%%%%%%%%%%%%%%%%%%%%%%%%%%%

%----------------------------------------------------------------------------------------
%	CLASS, PACKAGES AND OTHER DOCUMENT CONFIGURATIONS
%----------------------------------------------------------------------------------------

\documentclass[
	aspectratio=169,
	t,
	onlytextwidth,
	10pt,
]{beamer}

\usetheme{Imperial}
\useoutertheme{miniframes}

% Override monospaced font to support Unicode math symbols in code
% (e.g., ∀, ∃, →) that appear in Lean snippets
% Reset fontspec Path so system fonts can be found
\defaultfontfeatures{Ligatures=TeX,Path=}
\IfFontExistsTF{FreeMono}{
  \setmonofont{FreeMono}[Scale=MatchLowercase]
}{
  \setmonofont{Imperial Sans Text}[Scale=MatchLowercase]
}

%----------------------------------------------------------------------------------------
%	REQUIRED PACKAGES FOR CODE AND MATH
%----------------------------------------------------------------------------------------

% Code highlighting with minted for Lean4
\usepackage{minted}
\usemintedstyle{pastie}
\setminted{
	fontsize=\small,
	linenos=true,
	python3=true,
	breaklines=true,
	frame=lines,
}

% Math packages
\usepackage{amsmath}
\usepackage{amssymb}
\usepackage{mathtools}

% Additional utilities
\usepackage{booktabs}
\usepackage{tabularx}
\usepackage{changepage}

% For code snippets
% (listings removed; minted is used for Lean code blocks)

%----------------------------------------------------------------------------------------
%	CUSTOM COMMANDS FOR PRESENTATIONS
%----------------------------------------------------------------------------------------

% GitHub URL helper for code references
\newcommand{\githuburl}[4]{%
	\href{https://github.com/FrankieeW/GroupAction/blob/#1#2\#L#3-L#4}{\texttt{#2:#3-#4}}%
}

% Per-line GitHub linking for minted code blocks
\newcommand{\leancodeurl}{}
\newcommand{\setleancodeurl}[1]{\renewcommand{\leancodeurl}{#1}}
\renewcommand{\theFancyVerbLine}{\arabic{FancyVerbLine}}
\renewcommand{\FancyVerbFormatLine}[1]{\href{\leancodeurl\#L\arabic{FancyVerbLine}}{#1}}
\newcommand{\leancodefile}[4]{%
	\setleancodeurl{#3}%
	\inputminted[fontsize=\tiny,linenos=true,firstline=#1,lastline=#2,firstnumber=#1]{lean}{#4}%
}

% Lean code inline (safe for underscores and unicode)
\newcommand{\leancode}[1]{\texttt{\small\detokenize{#1}}}

% Two-column layout helper (math vs code)
\newcommand{\mathvscode}[4]{
	\begin{columns}[T]
		\begin{column}{0.48\linewidth}
			\textbf{Mathematics}\\[0.3em]
			#1
		\end{column}
		\begin{column}{0.48\linewidth}
			\textbf{Lean 4 Code}\\[0.3em]
			#2
		\end{column}
	\end{columns}
}

% Theorem definition
\newcommand{\thmbox}[2]{
	\fbox{\begin{minipage}{\dimexpr\linewidth-2\fboxsep-2\fboxrule}
	\textbf{#1}\\[0.5em]
	#2
	\end{minipage}}
}

%----------------------------------------------------------------------------------------
%	PRESENTATION METADATA
%----------------------------------------------------------------------------------------

\title{Formalizing Group Action in Lean}
\subtitle{Project 1: Group Actions in Lean 4}
\author{Frankie Feng-Cheng WANG}
\institute{Department of Mathematics, Imperial College London}
\date{MATH70040 Formalising Mathematics}

% Additional metadata for title slide
\newcommand{\course}{MATH70040 Formalising Mathematics}
\newcommand{\githublink}{https://github.com/FrankieeW/GroupAction}

%----------------------------------------------------------------------------------------
%	DOCUMENT START
%----------------------------------------------------------------------------------------

\begin{document}

%----------------------------------------------------------------------------------------
%	TITLE SLIDE (ICLBlue background)
%----------------------------------------------------------------------------------------

\begingroup
\setbeamercolor{background canvas}{bg=ICLBlue}
\setbeamercolor{title page title}{fg=white}
\setbeamercolor{title page subtitle}{fg=white}
\setbeamercolor{author}{fg=white}
\setbeamercolor{inst}{fg=white}
\setbeamercolor{date}{fg=white}
\setbeamertemplate{title page}[logo]{ICL_Logo_White.pdf}
\frame[plain, s]{\titlepage}
\endgroup

%----------------------------------------------------------------------------------------
%	AGENDA SLIDE (AUTO-GENERATED)
%----------------------------------------------------------------------------------------

\begingroup
\setbeamercolor{background canvas}{bg=ICLBlue}
\setbeamercolor{normal text}{fg=white}\usebeamercolor[fg]{normal text}
\setbeamercolor{page number in head/foot}{fg=white}

\begin{frame}
	\frametitle{\textcolor{white}{Outline}}
	\begin{columns}[T]
		\begin{column}{0.95\linewidth}
			\Large
			\vspace{0.1em}
			\textbf{1. Introduction}
			\quad Motivation \& Main Results\\[0.8em]

			\textbf{2. Core Definitions}
			\quad GroupAction, Faithful, Transitive\\[0.8em]

			\textbf{3. Examples}
			\quad Symmetric Group, $D_4$, and more\\[0.8em]

			\textbf{4. Key Theorems}\\
			\quad Theorem 16.3 (Permutation Representation) \\
			\quad Theorem 16.12 (Stabilizer Subgroup)\\[0.8em]

			\textbf{5. Reflection}
			\quad Challenges \& Future Extensions
		\end{column}
	\end{columns}
\end{frame}
\endgroup

% or
% \begin{frame}
% 	\frametitle{Contents}
% 	\tableofcontents[hideallsubsections]
% \end{frame}
%----------------------------------------------------------------------------------------
%	SECTION 1: INTRODUCTION (2 slides)
%----------------------------------------------------------------------------------------

\section{Introduction}

\begin{frame}
	\frametitle{What is a Group Action?}

	\textbf{Motivation:}
	\begin{itemize}
		\item Groups act on sets, formalizing \textit{symmetry}
		\item Connects abstract algebra with concrete transformations
		\item Foundation for orbit-stabilizer theorems, Burnside's lemma, etc.
	\end{itemize}

	\vspace{1em}

	\textbf{Project Goals:}
	\begin{itemize}
		\item[\checkmark] Core Definitions: \leancode{GroupAction} typeclass, Faithful, Transitive
		\item[\checkmark] Concrete Examples: Symmetric group, $D_4$, conjugation, etc.
		\item[\checkmark] Key Theorems: Permutation representation (Thm 16.3), Stabilizer subgroup (Thm 16.12)
	\end{itemize}

\end{frame}

%------------------------------------------------

\begin{frame}
	\frametitle{Main Results}

	\vspace{0.5em}

	\thmbox{Theorem 16.3: Permutation Representation}{%
		Every group action induces a group homomorphism $\phi: G \to \mathrm{Sym}(X)$
	}

	\vspace{1.5em}

	\thmbox{Theorem 16.12: Stabilizer Subgroup}{%
		For any $x \in X$, the stabilizer $G_x = \{g \in G \mid g \cdot x = x\}$ is a subgroup of $G$
	}

	\vspace{1.5em}

	\small\textit{Both theorems are fully formalized in Lean 4 with explicit proofs.}

\end{frame}

%----------------------------------------------------------------------------------------
%	SECTION 2: DEFINITIONS (3 slides)
%----------------------------------------------------------------------------------------

\section{Definitions}

\begin{frame}[fragile]
	\frametitle{Group Action Definition}
	\framesubtitle{Mathematical Foundation}

	\textbf{Mathematics}\\[0.5em]

	A group $G$ \textbf{acts} on a set $X$ via a function
	\[ \cdot: G \times X \to X \]

	\textbf{Axioms:}
	\begin{enumerate}
		\item \textbf{Associativity}: $(g_1 g_2) \cdot x = g_1 \cdot (g_2 \cdot x)$
		\item \textbf{Identity}: $1 \cdot x = x$
	\end{enumerate}

	\vspace{0.3em}
	for all $g_1, g_2 \in G, x \in X$

	\vspace{1em}

	\textbf{Lean 4 Code}\\[0.5em]
	\leancodefile{16}{22}{https://github.com/FrankieeW/GroupAction/blob/v1.2.1-lean-only/lean/GroupAction/Defs.lean}{../Lean/GroupAction/Defs.lean}

\end{frame}

%------------------------------------------------

\begin{frame}[fragile]
	\frametitle{Orbits and Stabilizers}

	% TODO: Define orbit and stabilizer sets
	% - Orbit: Orb(x) = {g · x | g ∈ G}
	% - Stabilizer: Stab(x) = {g ∈ G | g · x = x}
	% - Lean code for these definitions
	% - Connection to equivalence classes

\end{frame}

%------------------------------------------------

\begin{frame}[fragile]
	\frametitle{Faithful Actions}

	\textbf{Mathematics}\\[0.5em]

	An action is \textbf{faithful} if distinct group elements act differently.

	\vspace{0.8em}

	\textbf{Formally:}
	\[ \forall g_1, g_2 \in G, \, (\forall x, g_1 \cdot x = g_2 \cdot x) \Rightarrow g_1 = g_2 \]

	\vspace{0.8em}

	\textbf{Intuition:} The action ``faithfully represents'' the group structure

	\vspace{1em}

	\textbf{Lean 4 Code}\\[0.5em]
	\leancodefile{28}{29}{https://github.com/FrankieeW/GroupAction/blob/v1.2.1-lean-only/lean/GroupAction/Defs.lean}{../Lean/GroupAction/Defs.lean}

	\vspace{0.3em}
	{\tiny Source: \githuburl{v1.2.1-lean-only}{/Defs.lean}{28}{29}}

\end{frame}

%------------------------------------------------

\begin{frame}[fragile]
	\frametitle{Transitive Actions}

	\textbf{Mathematics}\\[0.5em]

	An action is \textbf{transitive} if any element can be moved to any other.

	\vspace{0.8em}

	\textbf{Formally:}
	\[ \forall x_1, x_2 \in X, \exists g \in G, \, g \cdot x_1 = x_2 \]

	\vspace{0.8em}

	\textbf{Intuition:} The group ``acts transitively'' on the entire set

	\vspace{1em}

	\textbf{Lean 4 Code}\\[0.5em]
	\leancodefile{36}{37}{https://github.com/FrankieeW/GroupAction/blob/v1.2.1-lean-only/lean/GroupAction/Defs.lean}{../Lean/GroupAction/Defs.lean}

	\vspace{0.3em}
	{\tiny Source: \githuburl{v1.2.1-lean-only}{/Defs.lean}{36}{37}}

\end{frame}

%----------------------------------------------------------------------------------------
%	SECTION 3: EXAMPLES (7 slides)
%----------------------------------------------------------------------------------------

\section{Examples}

\begin{frame}[fragile]
	\frametitle{Example 1: Symmetric Group on $X$}

	$\mathrm{Sym}(X)$ acts on $X$ by applying permutations.

	\vspace{0.8em}

	\leancodefile{21}{28}{https://github.com/FrankieeW/GroupAction/blob/v1.2.1-lean-only/lean/GroupAction/Examples.lean}{../Lean/GroupAction/Examples.lean}

	\vspace{0.8em}

	\textit{One of the most fundamental actions.}

	{\tiny Source: \githuburl{v1.2.1-lean-only}{/Examples.lean}{21}{28}}

\end{frame}

%------------------------------------------------

\begin{frame}[fragile]
	\frametitle{Example 1: Symmetric Group Properties}
	\framesubtitle{Faithful \& Transitive}

	\begin{columns}[T]
		\begin{column}{0.48\linewidth}
			\textbf{Faithful}\\[0.3em]

			\leancodefile{30}{36}{https://github.com/FrankieeW/GroupAction/blob/v1.2.1-lean-only/lean/GroupAction/Examples.lean}{../Lean/GroupAction/Examples.lean}
		\end{column}
		\begin{column}{0.48\linewidth}
			\textbf{Transitive}\\[0.3em]

			\leancodefile{37}{43}{https://github.com/FrankieeW/GroupAction/blob/v1.2.1-lean-only/lean/GroupAction/Examples.lean}{../Lean/GroupAction/Examples.lean}
		\end{column}
	\end{columns}

	{\tiny Source: \githuburl{v1.2.1-lean-only}{/Examples.lean}{30}{43}}

\end{frame}

%------------------------------------------------

\begin{frame}[fragile]
	\frametitle{Example 2: Left Multiplication}

	$G$ acts on itself by left multiplication: $g_1 \cdot g_2 = g_1 * g_2$

	\vspace{0.8em}

	\leancodefile{47}{54}{https://github.com/FrankieeW/GroupAction/blob/v1.2.1-lean-only/lean/GroupAction/Examples.lean}{../Lean/GroupAction/Examples.lean}

	\vspace{0.8em}

	\small This action is \textbf{transitive} but \textbf{not faithful}.

	{\tiny Source: \githuburl{v1.2.1-lean-only}{/Examples.lean}{47}{54}}

\end{frame}

%------------------------------------------------

\begin{frame}[fragile]
	\frametitle{Example 3: Subgroup Action}

	A subgroup $H \leq G$ acts on $G$ by left multiplication.

	\vspace{0.8em}

	\leancodefile{59}{66}{https://github.com/FrankieeW/GroupAction/blob/v1.2.1-lean-only/lean/GroupAction/Examples.lean}{../Lean/GroupAction/Examples.lean}

	\vspace{0.8em}

	\small Coercion from \leancode{H} to \leancode{G} handled implicitly.

		{\tiny Source: \githuburl{v1.2.1-lean-only}{/Examples.lean}{59}{66}}

\end{frame}

%------------------------------------------------

\begin{frame}[fragile]
	\frametitle{Example 4: Conjugation}

	Conjugation: $h \cdot g = h g h^{-1}$

	\vspace{0.6em}

	\leancodefile{69}{79}{https://github.com/FrankieeW/GroupAction/blob/v1.2.1-lean-only/lean/GroupAction/Examples.lean}{../Lean/GroupAction/Examples.lean}

	\vspace{0.6em}

	\small Orbits are \textbf{conjugacy classes}; stabilizers are \textbf{centralizers}.

	{\tiny Source: \githuburl{v1.2.1-lean-only}{/Examples.lean}{69}{79}}

\end{frame}

%------------------------------------------------

\begin{frame}[fragile]
	\frametitle{Example 5: Scalar Action on $\mathbb{C}^n$}

	$\mathbb{C}^\times$ acts on $\mathbb{C}^n$ by componentwise multiplication.

	\vspace{0.6em}

	\leancodefile{84}{93}{https://github.com/FrankieeW/GroupAction/blob/v1.2.1-lean-only/lean/GroupAction/Examples.lean}{../Lean/GroupAction/Examples.lean}

	\vspace{0.6em}

	\small Uses \leancode{Fin n → ℂ} to represent $\mathbb{C}^n$ in Lean.

		{\tiny Source: \githuburl{v1.2.1-lean-only}{/Examples.lean}{84}{93}}

\end{frame}

%------------------------------------------------

\begin{frame}[fragile]
	\frametitle{Example 6: Dihedral Group $D_4$}

	$D_4$ (symmetries of a square) acts on $\mathbb{Z}/4$ (vertices).

	\vspace{0.4em}

	\leancodefile{130}{160}{https://github.com/FrankieeW/GroupAction/blob/v1.2.1-lean-only/lean/GroupAction/Examples.lean}{../Lean/GroupAction/Examples.lean}

	\vspace{0.4em}

	\small \textbf{Geometric interpretation}: Rotations and reflections of square vertices.

		{\tiny Source: \githuburl{v1.2.1-lean-only}{/Examples.lean}{130}{160}}

\end{frame}

%----------------------------------------------------------------------------------------
%	SECTION 4: ORBITS & STABILIZERS (2 slides)
%----------------------------------------------------------------------------------------

\section{Orbits and Stabilizers}

\begin{frame}[fragile]
	\frametitle{Orbit-Stabilizer Theorem Setup}
	\framesubtitle{Relating |G|, |Orb(x)|, |Stab(x)|}

	% TODO: State and explain the theorem
	% - Theorem 16.3: |G| = |Orb(x)| · |Stab(x)|
	% - Bijection between Stab(x) and G/Stab(x)
	% - Lean type-theoretic formulation

\end{frame}

%------------------------------------------------

\begin{frame}[fragile]
	\frametitle{Burnside's Lemma Application}
	\framesubtitle{Counting Fixed Points}

	% TODO: Number of orbits formula
	% - Formula: # orbits = (1/|G|) Σ_{g∈G} |Fix(g)|
	% - Symmetry counting problem
	% - Lean proof sketch or reference

\end{frame}

%----------------------------------------------------------------------------------------
%	SECTION 5: THEOREM 16.3 - Orbit-Stabilizer (5 slides)
%----------------------------------------------------------------------------------------

\section{Theorem 16.3}

\begin{frame}[fragile]
	\frametitle{Theorem 16.3: Permutation Representation}

	\begin{block}{Theorem 16.3}
		Every group action induces a group homomorphism $\phi: G \to \mathrm{Sym}(X)$ such that:
		\[ \phi(g)(x) = g \cdot x \quad \text{for all } g \in G, x \in X \]
	\end{block}

	\vspace{1em}

	\textit{This theorem connects group actions with permutation representations.}

\end{frame}

%------------------------------------------------

\begin{frame}[fragile]
	\frametitle{Proof Strategy: 5 Steps}

	\begin{enumerate}
		\item \textbf{Define $\sigma_g$}: For each $g \in G$, construct $\sigma_g: X \to X$

		\item \textbf{Prove Bijection}: Show $\sigma_g$ is bijective (left/right inverse)

		\item \textbf{Construct $\phi$}: Package $\sigma_g$ as $\phi(g) \in \mathrm{Sym}(X)$

		\item \textbf{Verify Homomorphism}: Prove $\phi(g_1 g_2) = \phi(g_1) \circ \phi(g_2)$

		\item \textbf{Package Theorem}: Combine all pieces into final statement
	\end{enumerate}

\end{frame}

%------------------------------------------------

\begin{frame}[fragile]
	\frametitle{Step 1-2: Define $\sigma_g$ \& Prove Bijection}

	\textbf{Define $\sigma_g$}\\[0.5em]

	\leancodefile{27}{29}{https://github.com/FrankieeW/GroupAction/blob/v1.2.1-lean-only/lean/GroupAction/Permutation.lean}{../Lean/GroupAction/Permutation.lean}

	\vspace{1em}

	\textbf{Prove Bijective}\\[0.5em]

	\leancodefile{34}{64}{https://github.com/FrankieeW/GroupAction/blob/v1.2.1-lean-only/lean/GroupAction/Permutation.lean}{../Lean/GroupAction/Permutation.lean}

	\vspace{0.8em}

	\small \textbf{Key insight:} $g^{-1}$ provides the inverse map.

		{\tiny Source: \githuburl{v1.2.1-lean-only}{/Permutation.lean}{27}{64}}

\end{frame}

%------------------------------------------------

\begin{frame}[fragile]
	\frametitle{Step 3-4: Construct $\phi$ \& Verify Homomorphism}

	\textbf{Define $\phi: G \to \mathrm{Sym}(X)$}\\[0.5em]

	\leancodefile{69}{70}{https://github.com/FrankieeW/GroupAction/blob/v1.2.1-lean-only/lean/GroupAction/Permutation.lean}{../Lean/GroupAction/Permutation.lean}

	\vspace{1em}

	\textbf{Prove $\phi(g_1 g_2) = \phi(g_1) \circ \phi(g_2)$}\\[0.5em]

	\leancodefile{81}{90}{https://github.com/FrankieeW/GroupAction/blob/v1.2.1-lean-only/lean/GroupAction/Permutation.lean}{../Lean/GroupAction/Permutation.lean}

	\vspace{0.8em}

	\small \textbf{Highlight:} Uses \leancode{GroupAction.ga_mul} axiom for associativity.

		{\tiny Source: \githuburl{v1.2.1-lean-only}{/Permutation.lean}{69}{90}}

\end{frame}

%------------------------------------------------

\begin{frame}[fragile]
	\frametitle{Step 5: Package the Theorem}

	\textbf{Final Lean Theorem}\\[0.8em]

	\leancodefile{102}{108}{https://github.com/FrankieeW/GroupAction/blob/v1.2.1-lean-only/lean/GroupAction/Permutation.lean}{../Lean/GroupAction/Permutation.lean}

	\vspace{1em}

	\textbf{Proof by construction:} We exhibit $\phi$ and verify all properties.
	\begin{itemize}
		\item Action property: $\phi(g)(x) = g \cdot x$
		\item Homomorphism property: $\phi(g_1 g_2) = \phi(g_1) \circ \phi(g_2)$
		\item Identity property: $\phi(1) = \text{id}$
	\end{itemize}

	{\tiny Source: \githuburl{v1.2.1-lean-only}{/Permutation.lean}{102}{108}}

\end{frame}

%----------------------------------------------------------------------------------------
%	SECTION 6: THEOREM 16.12 (2 slides)
%----------------------------------------------------------------------------------------

\section{Theorem 16.12}

\begin{frame}[fragile]
	\frametitle{Theorem 16.12: Stabilizer Subgroup}

	\textbf{Mathematics}\\[0.5em]

	\textbf{Definition:} The \textbf{stabilizer} of $x \in X$ is:
	\[ G_x = \{g \in G \mid g \cdot x = x\} \]

	\vspace{0.8em}

	\textbf{Theorem 16.12:} $G_x$ is a subgroup of $G$

	\vspace{0.6em}

	\textbf{Proof requires:}
	\begin{enumerate}
		\item $1 \in G_x$ (identity)
		\item $g_1, g_2 \in G_x \Rightarrow g_1 g_2 \in G_x$ (closure)
		\item $g \in G_x \Rightarrow g^{-1} \in G_x$ (inverses)
	\end{enumerate}

	\vspace{1em}

	\textbf{Lean Structure}\\[0.5em]

	\leancodefile{25}{33}{https://github.com/FrankieeW/GroupAction/blob/v1.2.1-lean-only/lean/GroupAction/Stabilizer.lean}{../Lean/GroupAction/Stabilizer.lean}

	\vspace{0.5em}

	\small Lean requires explicit proofs of all three subgroup axioms.

		{\tiny Source: \githuburl{v1.2.1-lean-only}{/Stabilizer.lean}{25}{33}}

\end{frame}

%------------------------------------------------

\begin{frame}[fragile]
	\frametitle{Stabilizer in Lean: Definition \& Proof}

	\textbf{Part 1: Define Stabilizer Set} (Stabilizer.lean:25-26)\\[0.5em]

	\leancodefile{25}{26}{https://github.com/FrankieeW/GroupAction/blob/v1.2.1-lean-only/lean/GroupAction/Stabilizer.lean}{../Lean/GroupAction/Stabilizer.lean}

	\vspace{1em}

	\textbf{Part 2: Construct Subgroup} (Stabilizer.lean:29-53, key excerpts)\\[0.5em]

	\leancodefile{29}{53}{https://github.com/FrankieeW/GroupAction/blob/v1.2.1-lean-only/lean/GroupAction/Stabilizer.lean}{../Lean/GroupAction/Stabilizer.lean}

	\vspace{0.8em}

	\small \textbf{Highlight:} Each subgroup axiom proven with \leancode{calc} proofs.

		{\tiny Source: \githuburl{v1.2.1-lean-only}{/Stabilizer.lean}{25}{53}}

\end{frame}

%----------------------------------------------------------------------------------------
%	SECTION 7: REFLECTION (3 slides)
%----------------------------------------------------------------------------------------

\section{Reflection}

\begin{frame}
	\frametitle{What Lean Guarantees}

	\begin{itemize}
		\item[\checkmark] \textbf{Type Correctness}: All functions type-check, no runtime type errors

		\item[\checkmark] \textbf{No Hidden Assumptions}: Every axiom explicitly declared

		\item[\checkmark] \textbf{Axiom Matching}: Our proofs use only standard mathlib axioms (no choice beyond mathlib)

		\item[\checkmark] \textbf{Subgroup Verification}: \leancode{Stabilizer} is proven to satisfy all subgroup axioms

		\item[\checkmark] \textbf{Homomorphism Properties}: $\phi$ proven to preserve group structure
	\end{itemize}

	\vspace{1.5em}

	\textit{Lean's proof assistant guarantees mathematical correctness—no gaps, no handwaving.}

\end{frame}

%------------------------------------------------

\begin{frame}
	\frametitle{Challenges \& Lessons Learned}

	\textbf{Technical Challenges:}
	\begin{itemize}
		\item \textbf{Typeclass Resolution}: Manual instance declarations for \leancode{GroupAction}
		\item \textbf{Equiv Mechanism}: Understanding \leancode{Equiv.Perm} vs raw bijections
		\item \textbf{Coercions}: Handling coercion from \leancode{Subgroup H} to \leancode{H} in examples
		\item \textbf{Function Extensionality}: Using \leancode{Equiv.ext} to prove permutation equality
	\end{itemize}

	\vspace{1em}

	\textbf{Lessons:}
	\begin{itemize}
		\item Read Mathlib source code for patterns
		\item Use \leancode{calc} mode for clarity
		\item Lean enforces rigor: every step must be justified
	\end{itemize}

\end{frame}

%------------------------------------------------

\begin{frame}
	\frametitle{Conclusion \& Future Work}

	\textbf{What We Achieved:}
	\begin{itemize}
		\item Formalized core group action theory in Lean 4
		\item 7 concrete examples from abstract algebra
		\item 2 fundamental theorems with complete proofs
	\end{itemize}

	\vspace{1.5em}

	\textbf{Future Extensions:}
	\begin{itemize}
		\item \textbf{Orbit-Stabilizer Theorem}: $|G| = |\text{Orb}(x)| \cdot |G_x|$
		\item \textbf{Burnside's Lemma}: Counting orbits under symmetry
		\item \textbf{Applications}: Cayley's Theorem, Sylow Theorems
		\item \textbf{Category Theory}: Generalizations to functors and natural transformations
	\end{itemize}

\end{frame}

%----------------------------------------------------------------------------------------
%	THANK YOU SLIDE
%----------------------------------------------------------------------------------------

\begingroup
\setbeamercolor{background canvas}{bg=ICLBlue}
\setbeamercolor{normal text}{fg=white}\usebeamercolor[fg]{normal text}
\setbeamercolor{page number in head/foot}{fg=white}

\begin{frame}[plain]
	\vfill
	\centering

	{\Huge\textbf{Thank You!}}

	\vspace{2em}

	{\large\textit{Questions?}}

	\vspace{2em}

	\hrule

	\vspace{2em}

	{\large\textbf{GitHub Repository}}\\[0.5em]
	\small\url{https://github.com/FrankieeW/GroupAction}\\
	Version: \texttt{v1.2.1-lean-only} (stable)

	\vspace{2em}

	{\large\textbf{Contact}}\\[0.5em]
	\small Frankie Feng-Cheng WANG\\
	Department of Mathematics, Imperial College London\\
	MATH70040 Formalising Mathematics

	\vfill
\end{frame}
\endgroup

%----------------------------------------------------------------------------------------

\end{document}
